\chapter{Technical Details}
\section{Cross References}
\label{sec:CrossReferences}
There are usually a lot of cross references in scientific texts or in a thesis. Typical entities referred to in the text are:
\begin{description}
	\item [sections] -- for example section \ref{sec:ExpeditaDistinctio}. If we refer to a section that is very far from the current page, it is usual to include the corresponding page number with the section number, such as section \ref{sec:Introduction} on page \pageref{sec:Introduction}.
	\item [figures] -- for example figures \ref{fig:WritingThesis}, \ref{fig:CoffeAndComputerInAppendix}, and \ref{fig:TSquareFractal}. We can also refer to high level figures, e.g.\ figure \ref{fig:TopLevelFigureLabel}, which are divided into separate subfigures such as \ref{fig:Subfig1} and \ref{fig:Subfig2}.
	\item [tables] -- for example tables \ref{tab:ExpResults} and \ref{tab:Sidewaystable}. There is high level table \ref{tab:TopLevelTableLabel} and subtables \ref{tab:Subtable1} and \ref{tab:Subtable2} too.
	\item [equations] -- equation numbers are usually enclosed by parentheses, such as equation (\ref{eq:A}), (\ref{eq:B}) or (\ref{eq:C}).
	\item [source code listings] -- for example listing \ref{src:CppListing}. The listing \ref{src:PythonListing} is an example of listing in different language, in this case Python, than the default C++. We can also refer to long listing, such as listing \ref{src:CppExternal} on page \pageref{src:CppExternal} in appendix \ref{sec:Appendix1}, which is loaded form external source code file.
\end{description}

\section{How to cite}
\label{sec:HowToCite}
\subsection{In-text citing}
It is not necessary to mention an author's name, pages used, or date of publication in the in-text citation. Instead, refer to the source with a number in a square bracket, e.g. [1], that will then correspond to the full citation in your reference list. For example we can cite resources like \emph{articles} \cite{herrmann, bertram, moore, yoon, sigfridsson, baez/article}, \emph{books} \cite{wilde, nietzsche:ksa1, averroes/bland, hammond, cotton, knuth:ct:a, gerhardt, gonzalez, companion}, \emph{periodicals} \cite{jcg}, \emph{theses} \cite{geer}, \emph{patents} \cite{kowalik, almendro, sorace, laufenberg}, \emph{online resources} \cite{ctan, wassenberg, itzhaki, markey, baez/online}, \emph{manuals} \cite{cms}.

\subsection{Creating a Reference List}
The Reference List appears at the end of your thesis and provides the full citations for all the references you have used.  

\section{How to compile}
To build this thesis demo from scratch you have to run pdf\LaTeX{} and Biber several times in following way:
\begin{verbatim}
pdflatex <main file name>
biber <main file name>
pdflatex <main file name>
pdflatex <main file name>
pdflatex <main file name>
\end{verbatim}
\endinput